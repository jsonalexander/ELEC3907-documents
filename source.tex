\documentclass{article}

\usepackage[margin=2in]{geometry}

\usepackage{float}
\usepackage{caption}
\usepackage{wrapfig}

\usepackage{hyperref}
\usepackage{cleveref}

\usepackage{graphicx}

\begin{document}
	
	\begin{titlepage}\centering

	{\huge ELEC 3907 Group 2H}

	\vspace*{1em}\hrule\vspace*{1em}

	{\huge\bf Project Proposal: Arcade-Style Game Console}

	\vspace*{0.5em}

	\today

	\vspace*{1em}\hrule\vspace*{1em}

	\bigskip

	\begin{tabular}{lr}
		Jason Alexander & 101309867 \\
		Jason Alexander & 101309867 \\
		Jason Alexander & 101309867 \\
		Jason Alexander & 101309867 \\
		Jason Alexander & 101309867 \\
	\end{tabular}

	\vfil

	\tableofcontents

\end{titlepage}


	\newgeometry{margin=1in}

	\section{Project Overview}

		The goal of this project is to provide a compact arcade-style game console designed to restore hands-on interaction commonly found in traditional arcade systems.
		The console uses a 2D LED grid as the primary display as well as physical controls and audio.
		It is designed to run multiple classic games such as Tron, Battleship, and Snake using a multi-purpose hardware layout and modular software design.

	\section{Design}

		\subsection{System-Level Description}

			The console is a game system where three different games can be played using a physical display and tactile controls.
			The console does not require an internet connection and can be played anywhere there is a source of power.
			It provides a compact, physical, and portable version of several arcade games such that people of all ages can enjoy using the simplistic controls and peripherals.
			Additionally, more games may be uploaded and included, allowing for different people with different preferences to enjoy the console.

			\subsubsection{The Hardware}

				\begin{wrapfigure}{r}{0.5\textwidth}
					\centering
					\includegraphics[width=0.48\textwidth]{img/ortho-view.png}
					\caption{Orthographic views of the hardware design}
					\label{fig:hardware-ortho}
				\end{wrapfigure}

				The console itself (seen in \cref{fig:hardware-ortho}) consists of multiple subsystems, including a 16x16 LED display with a divider, where each of the 256 pixels of the display can be individually addressed to display an almost arbitrary number of colours.
				There are two LCD screens for game selection, acting as a menu for players to interface with the console.
				These LCD screens will also act as instructional screens to guide players through the games.
				There are two removable and adjustable controllers, where the controller can be plugged into either side of the console to accommodate right- or left-handedness or for single and multiplayer options.
				The speaker provides sound effects for audio feedback and soundscapes during gameplay.
				There is also a freestanding divider that will be used during games that require the players to not see each others' boards during gameplay---which is the case in battleship, for example.
				The microcontroller is programmed with the game logic, interfacing with the other components to provide a seamless experience while using the console.
				Finally, the power supply powers the display and the microcontroller, the latter of which subsequently powers the other components.

			\subsubsection{The Software}

				The software system must be able to interface with all the components outlined, taking input from the controllers, controlling the LED matrix, using the LCD displays to relay information to each player, and playing audio.
				The system is constrained by the memory limitations of each microcontroller, as well as their single-threadedness.
				To overcome scheduling issues, especially with continuous-delivery applications like audio playback, a second microcontroller will be dedicated to receiving playback signals from the main microcontroller and controlling the speaker continuously.
				This second microcontroller can be much less powerful than the main one, essentially serving as an asynchronous processing thread to avoid audio cut-outs and maintain responsiveness of the console proper.

				Additionally, since the hardware system is designed to be general-purpose, we have decided to leverage object-oriented programming techniques in C++ to model the hardware system and allow for composability.
				The hardware-software interface is designed in such a way that basic operations like controlling the LED matrix can be inherited and extended to implement game-specific behaviours, reducing unnecessary boilerplate code and further memory usage.
				\cref{fig:system-diagram} provides a block diagram outlining the general control flow of the system.

				\begin{figure}[H]
					\centering
					\includegraphics[width=0.75\textwidth]{img/block-diagram.png}
					\caption{Block diagram of the system}
					\label{fig:system-diagram}
				\end{figure}

		\subsection{Preliminary Specifications and Requirements}

		\subsection{Design Approaches}

	\section{Risk and Safety Assessment}

	\section{Sustainability and Lifecycle}

	\section{Team Roles and Responsibilities}

	\section{Project Timing}

	\section{Conclusion}

\end{document}
